% 2-15-rb-tree.tex

%%%%%%%%%%%%%%%%%%%%
\documentclass[a4paper, justified]{tufte-handout}

\input{hw-preamble} % feel free to modify this file
%%%%%%%%%%%%%%%%%%%%
\title{第3-8讲: 单源最短路}
\me{马骏}{majun@nju.edu.cn}{}{}
\date{\zhtoday} % or like 2019年9月13日
%%%%%%%%%%%%%%%%%%%%
\begin{document}
\maketitle
%%%%%%%%%%%%%%%%%%%%
\noplagiarism % always keep this line
%%%%%%%%%%%%%%%%%%%%
\begin{abstract}
  % \begin{center}{\fcolorbox{blue}{yellow!60}{\parbox{0.65\textwidth}{\large 
  %   \begin{itemize}
  %     \item 
  %   \end{itemize}}}}
  % \end{center}
\end{abstract}
%%%%%%%%%%%%%%%%%%%%
\beginrequired

%%%%%%%%%%%%%%%
\begin{problem}[TC 24.1-2]
\end{problem}

\begin{solution}
\end{solution}
%%%%%%%%%%%%%%%

%%%%%%%%%%%%%%%
\begin{problem}[TC 24.1-3]
\end{problem}

\begin{solution}
\end{solution}
%%%%%%%%%%%%%%%

%%%%%%%%%%%%%%%
\begin{problem}[TC 24.1-4]
\end{problem}

\begin{solution}
\end{solution}
%%%%%%%%%%%%%%%

%%%%%%%%%%%%%%%
\begin{problem}[TC 24.2-2]
\end{problem}

\begin{solution}
\end{solution}
%%%%%%%%%%%%%%%

%%%%%%%%%%%%%%%
\begin{problem}[TC 24.3-2]
\end{problem}

\begin{solution}
\end{solution}
%%%%%%%%%%%%%%%

%%%%%%%%%%%%%%%
\begin{problem}[TC 24.3-4]
\end{problem}

\begin{solution}
\end{solution}
%%%%%%%%%%%%%%%


%%%%%%%%%%%%%%%
\begin{problem}[TC 24.3-7]
\end{problem}

\begin{solution}
\end{solution}
%%%%%%%%%%%%%%%

%%%%%%%%%%%%%%%
\begin{problem}[TC 24.5-2]
\end{problem}

\begin{solution}
\end{solution}
%%%%%%%%%%%%%%%

%%%%%%%%%%%%%%%
\begin{problem}[TC 24.5-5]
\end{problem}

\begin{solution}
\end{solution}
%%%%%%%%%%%%%%%

%%%%%%%%%%%%%%%
\begin{problem}[TC Problem 24-3]
\end{problem}

\begin{solution}
\end{solution}
%%%%%%%%%%%%%%%



%%%%%%%%%%%%%%%%%%%%
\beginoptional

%%%%%%%%%%%%%%%
\begin{problem}[TC Problem 24-2]
\end{problem}

\begin{solution}
\end{solution}
%%%%%%%%%%%%%%%


%%%%%%%%%%%%%%%%%%%%
\beginot
%%%%%%%%%%%%%%%
%%%%%%%%%%%%%%%
\begin{ot}[Delta stepping algorithm]

	\noindent 参考资料:
	\begin{itemize}
	\item \href{https://en.wikipedia.org/wiki/Parallel_single-source_shortest_path_algorithm}{https://en.wikipedia.org/wiki/Parallel\_single-source\_shortest\_path\_algorithm}
	\item 	\href{https://www.sciencedirect.com/science/article/pii/S0196677403000762?via\%3Dihub} {Meyer, U.; Sanders, P. (2003-10-01). ``Δ-stepping: a parallelizable shortest path algorithm''. }
	\end{itemize}
	
	
\end{ot}

% \begin{solution}
% \end{solution}
%%%%%%%%%%%%%%%

\begin{ot}[Radius stepping algorithm]
	
	\noindent 参考资料:
	\begin{itemize}
		\item \href{https://en.wikipedia.org/wiki/Parallel_single-source_shortest_path_algorithm}{https://en.wikipedia.org/wiki/Parallel\_single-source\_shortest\_path\_algorithm}
		\item \href{https://dl.acm.org/doi/10.1145/2935764.2935765} {Blelloch, Guy E.; Gu, Yan; Sun, Yihan; Tangwongsan, Kanat (2016). "Parallel Shortest Paths Using Radius Stepping". Proceedings of the 28th ACM Symposium on Parallelism in Algorithms and Architectures - SPAA '16. New York, New York, USA: ACM Press: 443–454.}
	\end{itemize}
		
\end{ot}

% \begin{solution}
% \end{solution}
%%%%%%%%%%%%%%%




% \vspace{0.50cm}
%%%%%%%%%%%%%%%
% \begin{ot}[]
% 
%   \noindent 参考资料:
%   \begin{itemize}
%     \item 
%   \end{itemize}
% \end{ot}

% \begin{solution}
% \end{solution}
%%%%%%%%%%%%%%%

%%%%%%%%%%%%%%%%%%%%
% 如果没有需要订正的题目,可以把这部分删掉

% \begincorrection
%%%%%%%%%%%%%%%%%%%%

%%%%%%%%%%%%%%%%%%%%
% 如果没有反馈,可以把这部分删掉
\beginfb

% 你可以写
% ~\footnote{优先推荐 \href{problemoverflow.top}{ProblemOverflow}}:
% \begin{itemize}
%   \item 对课程及教师的建议与意见
%   \item 教材中不理解的内容
%   \item 希望深入了解的内容
%   \item $\cdots$
% \end{itemize}
%%%%%%%%%%%%%%%%%%%%
% \bibliography{2-5-solving-recurrence}
% \bibliographystyle{plainnat}
%%%%%%%%%%%%%%%%%%%%
\end{document}