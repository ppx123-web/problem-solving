% 2-15-rb-tree.tex

%%%%%%%%%%%%%%%%%%%%
\documentclass[a4paper, justified]{tufte-handout}

% hw-preamble.tex

% geometry for A4 paper
% See https://tex.stackexchange.com/a/119912/23098
\geometry{
  left=20.0mm,
  top=20.0mm,
  bottom=20.0mm,
  textwidth=130mm, % main text block
  marginparsep=5.0mm, % gutter between main text block and margin notes
  marginparwidth=50.0mm % width of margin notes
}

% for colors
\usepackage{xcolor} % usage: \color{red}{text}
% predefined colors
\newcommand{\red}[1]{\textcolor{red}{#1}} % usage: \red{text}
\newcommand{\blue}[1]{\textcolor{blue}{#1}}
\newcommand{\teal}[1]{\textcolor{teal}{#1}}

\usepackage{todonotes}

% heading
\usepackage{sectsty}
\setcounter{secnumdepth}{2}
\allsectionsfont{\centering\huge\rmfamily}

% for Chinese
\usepackage{xeCJK}
\usepackage{zhnumber}
\setCJKmainfont[BoldFont=FandolSong-Bold.otf]{FandolSong-Regular.otf}

% for fonts
\usepackage{fontspec}
\newcommand{\song}{\CJKfamily{song}} 
\newcommand{\kai}{\CJKfamily{kai}} 

% To fix the ``MakeTextLowerCase'' bug:
% See https://github.com/Tufte-LaTeX/tufte-latex/issues/64#issuecomment-78572017
% Set up the spacing using fontspec features
\renewcommand\allcapsspacing[1]{{\addfontfeature{LetterSpace=15}#1}}
\renewcommand\smallcapsspacing[1]{{\addfontfeature{LetterSpace=10}#1}}

% for url
\usepackage{hyperref}
\hypersetup{colorlinks = true, 
  linkcolor = teal,
  urlcolor  = teal,
  citecolor = blue,
  anchorcolor = blue}

\newcommand{\me}[4]{
    \author{
      {\bfseries 姓名:}\underline{#1}\hspace{2em}
      {\bfseries 学号:}\underline{#2}\hspace{2em}\\[10pt]
      {\bfseries 评分:}\underline{#3\hspace{3em}}\hspace{2em}
      {\bfseries 评阅:}\underline{#4\hspace{3em}}
  }
}

% Please ALWAYS Keep This.
\newcommand{\noplagiarism}{
  \begin{center}
    \fbox{\begin{tabular}{@{}c@{}}
      请独立完成作业,不得抄袭。\\
      若得到他人帮助, 请致谢。\\
      若参考了其它资料,请给出引用。\\
      鼓励讨论,但需独立书写解题过程。
    \end{tabular}}
  \end{center}
}

\newcommand{\goal}[1]{
  \begin{center}{\fcolorbox{blue}{yellow!60}{\parbox{0.50\textwidth}{\large 
    \begin{itemize}
      \item 体会``思维的乐趣''
      \item 初步了解递归与数学归纳法 
      \item 初步接触算法概念与问题下界概念
    \end{itemize}}}}
  \end{center}
}

% Each hw consists of four parts:
\newcommand{\beginrequired}{\hspace{5em}\section{作业 (必做部分)}}
\newcommand{\beginoptional}{\section{作业 (选做部分)}}
\newcommand{\beginot}{\section{Open Topics}}
\newcommand{\begincorrection}{\section{订正}}
\newcommand{\beginfb}{\section{反馈}}

% for math
\usepackage{amsmath, mathtools, amsfonts, amssymb}
\newcommand{\set}[1]{\{#1\}}

% define theorem-like environments
\usepackage[amsmath, thmmarks]{ntheorem}

\theoremstyle{break}
\theorempreskip{2.0\topsep}
\theorembodyfont{\song}
\theoremseparator{}
\newtheorem{problem}{题目}[subsection]
\renewcommand{\theproblem}{\arabic{problem}}
\newtheorem{ot}{Open Topics}

\theorempreskip{3.0\topsep}
\theoremheaderfont{\kai\bfseries}
\theoremseparator{:}
\theorempostwork{\bigskip\hrule}
\newtheorem*{solution}{解答}
\theorempostwork{\bigskip\hrule}
\newtheorem*{revision}{订正}

\theoremstyle{plain}
\newtheorem*{cause}{错因分析}
\newtheorem*{remark}{注}

\theoremstyle{break}
\theorempostwork{\bigskip\hrule}
\theoremsymbol{\ensuremath{\Box}}
\newtheorem*{proof}{证明}

% \newcommand{\ot}{\blue{\bf [OT]}}

% for figs
\renewcommand\figurename{图}
\renewcommand\tablename{表}

% for fig without caption: #1: width/size; #2: fig file
\newcommand{\fig}[2]{
  \begin{figure}[htbp]
    \centering
    \includegraphics[#1]{#2}
  \end{figure}
}
% for fig with caption: #1: width/size; #2: fig file; #3: caption
\newcommand{\figcap}[3]{
  \begin{figure}[htbp]
    \centering
    \includegraphics[#1]{#2}
    \caption{#3}
  \end{figure}
}
% for fig with both caption and label: #1: width/size; #2: fig file; #3: caption; #4: label
\newcommand{\figcaplbl}[4]{
  \begin{figure}[htbp]
    \centering
    \includegraphics[#1]{#2}
    \caption{#3}
    \label{#4}
  \end{figure}
}
% for margin fig without caption: #1: width/size; #2: fig file
\newcommand{\mfig}[2]{
  \begin{marginfigure}
    \centering
    \includegraphics[#1]{#2}
  \end{marginfigure}
}
% for margin fig with caption: #1: width/size; #2: fig file; #3: caption
\newcommand{\mfigcap}[3]{
  \begin{marginfigure}
    \centering
    \includegraphics[#1]{#2}
    \caption{#3}
  \end{marginfigure}
}

\usepackage{fancyvrb}

% for algorithms
\usepackage[]{algorithm}
\usepackage[]{algpseudocode} % noend
% See [Adjust the indentation whithin the algorithmicx-package when a line is broken](https://tex.stackexchange.com/a/68540/23098)
\newcommand{\algparbox}[1]{\parbox[t]{\dimexpr\linewidth-\algorithmicindent}{#1\strut}}
\newcommand{\hStatex}[0]{\vspace{5pt}}
\makeatletter
\newlength{\trianglerightwidth}
\settowidth{\trianglerightwidth}{$\triangleright$~}
\algnewcommand{\LineComment}[1]{\Statex \hskip\ALG@thistlm \(\triangleright\) #1}
\algnewcommand{\LineCommentCont}[1]{\Statex \hskip\ALG@thistlm%
  \parbox[t]{\dimexpr\linewidth-\ALG@thistlm}{\hangindent=\trianglerightwidth \hangafter=1 \strut$\triangleright$ #1\strut}}
\makeatother

% for footnote/marginnote
% see https://tex.stackexchange.com/a/133265/23098
\usepackage{tikz}
\newcommand{\circled}[1]{%
  \tikz[baseline=(char.base)]
  \node [draw, circle, inner sep = 0.5pt, font = \tiny, minimum size = 8pt] (char) {#1};
}
\renewcommand\thefootnote{\protect\circled{\arabic{footnote}}} % feel free to modify this file
%%%%%%%%%%%%%%%%%%%%
\title{第3-2讲: 贪心}
\me{赵超懿}{191870271}{}{}
\date{\zhtoday} % or like 2019年9月13日
%%%%%%%%%%%%%%%%%%%%
\begin{document}
\maketitle
%%%%%%%%%%%%%%%%%%%%
\noplagiarism % always keep this line
%%%%%%%%%%%%%%%%%%%%
\begin{abstract}
  % \begin{center}{\fcolorbox{blue}{yellow!60}{\parbox{0.65\textwidth}{\large 
  %   \begin{itemize}
  %     \item 
  %   \end{itemize}}}}
  % \end{center}
\end{abstract}
%%%%%%%%%%%%%%%%%%%%
\beginrequired

%%%%%%%%%%%%%%%
\begin{problem}[TC 16.1-2]
Suppose that instead of always selecting the first activity to finish, we instead select
the last activity to start that is compatible with all previously selected activities. Describe how this approach is a greedy algorithm, and prove that it yields an optimal
solution.
\end{problem}

\begin{solution}
首先,按照开始时间的循序排序(从小到大)\\
证明:对于非空子问题$S_k$,令$a_m$是$S_k$中开始时间最晚的活动,则$a_m$在$S_k$的某个最大兼容活动子集中。\\
设$a_j$是$A_k$($A_k$是$S_k$的最大兼容活动子集)中开始最晚的活动,若j=m,则证明,若$j\neq m$则选择开展$a_j$而不是$a_m$,(活动时间不会发生冲突)这时新的最大兼容活动子集与原来一样大,故$a_m$在$S_k$的某个最大兼容活动子集中。
\end{solution}
%%%%%%%%%%%%%%%

%%%%%%%%%%%%%%%
\begin{problem}[TC 16.1-3]
Not just any greedy approach to the activity-selection problem produces a maximum-size set of mutually compatible activities. Give an example to show that
the approach of selecting the activity of least duration from among those that are
compatible with previously selected activities does not work. Do the same for
the approaches of always selecting the compatible activity that overlaps the fewest
other remaining activities and always selecting the compatible remaining activity
with the earliest start time.

\end{problem}

\begin{solution}
例子:\\
1.时间0-7,以下分别对应事件1、2、3、4,该情况下最优解为选择事件1和4\\
$s_i$(事件开始时间)0 2 3 4\\
$f_i$(事件开始时间)4 5 5 7\\
当选择持续时间最短的事件,选择事件3,只能进行一个活动,不是最优解\\
2. 时间0-8,以下分别对应事件1、2、3、4,该情况下最优解为选择事件1、3\\
$s_i$(事件开始时间)0 3 4 0\\
$f_i$(事件开始时间)4 5 8 8\\
当选择与其他剩余活动重叠最少者:先选择事件3,不是最优解。\\
3. 时间0-7,以下分别对应事件1、2、3,该情况下最优解为选择事件2和3\\
$s_i$(事件开始时间)0 2 5\\
$f_i$(事件开始时间)7 5 7\\
当选择最早开始的事件,选择事件1,只能进行一个活动,不是最优解\\
\end{solution}
%%%%%%%%%%%%%%%

%%%%%%%%%%%%%%%
\begin{problem}[TC 16.2-1]
Prove that the fractional knapsack problem has the greedy-choice property.
\end{problem}

\begin{solution}
首先,问题一定有最优解。对物体按从高到低按单位价值排列,标记序号1,2,,,n\\
设装背包时是把要装的物体按单位价值从高到低装进背包,记物体的序号为$1,2,,,,k$\\
记$s_i$是把前i的物品装进背包的方案或把背包装满,若最优解在标号j个物品时(即没装物品中单位价值最高的)没有装,或没有装完(背包还没满),则将后面的装的物品换成标号为j的物体装进去,总价值高于原方案,故标号为j的物品一定在最优解中,故具有贪心选择性质。

\end{solution}
%%%%%%%%%%%%%%%

%%%%%%%%%%%%%%%
\begin{problem}[TC 16.2-2]
Give a dynamic-programming solution to the 0-1 knapsack problem that runs in O(nW) time, where n is the number of items and W is the maximum weight of
items that the thief can put in his knapsack.

\end{problem}

\begin{solution}
记v[i][j]为背包容量为j时,在前i个物品中选择装物体的价值最大值,初始化为0,s数组用来记录装的方案\\
0-1-knapack-problem\\
\hspace*{0.6 cm} for i = 1 to n\\
\hspace*{1.2 cm}   for j = 1 to W\\
\hspace*{1.8 cm}      if j >= w[i]\\
\hspace*{2.4 cm}         v[i][j]=v[i-1][j-w[i]];\\
\hspace*{2.4 cm}         s[i][j]=i;\\
\hspace*{0.6 cm} return v[n][W] and s;\\
\\
print(s,i,j)\\
\hspace*{0.6 cm} if i == 0   \\
\hspace*{1.2 cm}   return;\\
\hspace*{0.6 cm} output i;\\
\hspace*{0.6 cm} print(s,s[i][j]-1,j-w[i]);\\
\end{solution}
%%%%%%%%%%%%%%%

%%%%%%%%%%%%%%%
\begin{problem}[TC 16.3-2]

\end{problem}

\begin{solution}
\end{solution}
%%%%%%%%%%%%%%%

%%%%%%%%%%%%%%%
\begin{problem}[TC 16.3-5]
\end{problem}

\begin{solution}
\end{solution}
%%%%%%%%%%%%%%%

%%%%%%%%%%%%%%%
\begin{problem}[TC 16.3-8]
\end{problem}

\begin{solution}
\end{solution}
%%%%%%%%%%%%%%%


%%%%%%%%%%%%%%%
\begin{problem}[TC 17.1-3]
\end{problem}

\begin{solution}
\end{solution}
%%%%%%%%%%%%%%%

%%%%%%%%%%%%%%%
\begin{problem}[TC 17.2-2]
\end{problem}

\begin{solution}
\end{solution}
%%%%%%%%%%%%%%%

%%%%%%%%%%%%%%%
\begin{problem}[TC 17.4-1]
\end{problem}

\begin{solution}
\end{solution}
%%%%%%%%%%%%%%%

%%%%%%%%%%%%%%%%%%%%
\beginoptional

%%%%%%%%%%%%%%%
\begin{problem}[TC Problem 16-1 (Coin Changing)]
\end{problem}

\begin{solution}
\end{solution}
%%%%%%%%%%%%%%%

%%%%%%%%%%%%%%%%%%%%
\beginot
%%%%%%%%%%%%%%%
\begin{ot}[Ternary Disk]
	Trimedia Disks Inc. has developed ``ternary'' hard disks. 
	Each cell on a disk can now store values 0, 1, or 2 (instead of just 0 or 1). 
	
	To take advantage of this new technology, provide a modified Huffman
	algorithm for constructing an optimal variable-length prefix-free code for characters from an alphabet of size n, where the characters occur with known frequencies $f_1, f_2, \cdots , f_n$. 
	
	Prove that your algorithm is correct.
\end{ot}

% \begin{solution}
% \end{solution}
%%%%%%%%%%%%%%%

%%%%%%%%%%%%%%%
\begin{ot}[Intervals]
	
	Let $X$ be a set of $n$ intervals on the real line. 
	A subset of intervals $Y\subseteq X$ is called a \textit{\textbf{full path}} if the intervals in $Y$ cover the intervals in $X$, that is, any real value that is contained in some interval in $X$ is also contained in some interval in $Y$ . The \textit{size} of the full path is the number of intervals it contains.
	
	\mfigcap{width = 1.00\textwidth,height=1.5cm}{figs/intervals.png}{蓝色的7个区间组成一个完整路径(full path)}
	
	Describe and analyze a greedy algorithm to compute the smallest full path of $X$ as quickly as possible. 
	Assume that your input consists of two arrays $X_L \left[1..n\right]$ and $X_R \left[1..n\right]$, representing the left and right endpoints of the intervals in $X$. Don't forget to prove your greedy algorithm is correct!
\end{ot}

% \begin{solution}
% \end{solution}
%%%%%%%%%%%%%%%


% \vspace{0.50cm}
%%%%%%%%%%%%%%%
% \begin{ot}[]
% 
%   \noindent 参考资料:
%   \begin{itemize}
%     \item 
%   \end{itemize}
% \end{ot}

% \begin{solution}
% \end{solution}
%%%%%%%%%%%%%%%

%%%%%%%%%%%%%%%%%%%%
% 如果没有需要订正的题目,可以把这部分删掉

% \begincorrection
%%%%%%%%%%%%%%%%%%%%

%%%%%%%%%%%%%%%%%%%%
% 如果没有反馈,可以把这部分删掉
\beginfb

% 你可以写
% ~\footnote{优先推荐 \href{problemoverflow.top}{ProblemOverflow}}:
% \begin{itemize}
%   \item 对课程及教师的建议与意见
%   \item 教材中不理解的内容
%   \item 希望深入了解的内容
%   \item $\cdots$
% \end{itemize}
%%%%%%%%%%%%%%%%%%%%
% \bibliography{2-5-solving-recurrence}
% \bibliographystyle{plainnat}
%%%%%%%%%%%%%%%%%%%%
\end{document}